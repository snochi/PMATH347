\documentclass[pmath347]{subfiles}

%% ========================================================
%% document

\begin{document}

    \chap{Subgroups and Homomorphisms} 

    \section{Subgroups}
    
    \begin{definition}{Subgroup}{of a Group}
        Let $G$ be a group. $H\subseteq G$ is called a \emph{subgroup} if
        \begin{enumerate}
            \item for all $g,h\in H, gh\in H$;
            \item $e_G\in H$; and
            \item for all $g\in H, g^{-1} \in H$.
        \end{enumerate}
        We typically write $H\leq G$.
    \end{definition}

    \ex Consider $D_{2n}$, and let $s$ be the rotation. Then
    \begin{equation*}
        H = \left\lbrace s^k: k\in\left\lbrace 0,\ldots,n-1 \right\rbrace  \right\rbrace 
    \end{equation*}
    is a subgroup of $D_{2n}$. In fact, $H$ is the smallest subgroup containing $s$, denoted as $H=\left< s \right>$. 

    \begin{definition}{Trivial, Proper}{Subgroup}
        Let $G$ be a group.
        \begin{enumerate}
            \item $\left\lbrace e \right\rbrace$ is called the \emph{trivial subgroup}.\footnote{It is very clear why $\left\lbrace e \right\rbrace$ is indeed a subgroup of $G$.}
            \item A subgroup $H\leq G$ is called \emph{proper} if $H\neq G$. We write $H<G$.
        \end{enumerate}
    \end{definition}

    \begin{prop}{}
        Let $\left( G,\cdot \right)$ be a group. Then $\left( H,\cdot|_{H\times H} \right)$ is a group such that
        \begin{enumerate}
            \item $e_H=e_G$; and
            \item given any $g\in H$, the inverse of $g$ in $H$ is the same as the inverse of $g$ in $G$.
        \end{enumerate}
    \end{prop}

    \begin{proof}
        It is clear that $\cdot|_{H\times H}$ is a binary operation on $H$ by (a) of Def'n 2.1. For convenience, write $\star$ to denote $\cdot|_{H\times H}$. Since $\cdot$ is associative, $\star$ is associative. Also, it is clear that $e_G$ is identity for $\star$. Moreover, if $g\in H$, then its inverse $g^{-1} $ with respect to $\cdot$ is in $H$, where
        \begin{equation*}
            g\star g^{-1} = g^{-1} \star g = e_G.
        \end{equation*}
        But we know that $e_G=e_H$, $g^{-1} $ is the inverse of $g$ with respect to $\star$.
    \end{proof}

    \begin{definition}{Operation Induced}{on a Subgroup}
        Let $\left( G,\cdot \right)$ be a group and let $\left( H,\cdot|_{H\times H} \right)$ be a subgroup of $G$. Then $\cdot|_{H\times H}$ is called the \emph{operation induced} by $\cdot$ on $H$.
    \end{definition}

    \np[Induced Operations]We usually refer to $\cdot|_{H\times H}$ as $\cdot$.

    \clearpage
    \begin{prop}{Two-step Subgroup Test}
        Let $G$ be a group. Then $H\subseteq G$ is a subgroup of $G$ if and only if
        \begin{enumerate}
            \item $H\neq\emptyset$; and
            \item for every $g,h\in H, gh^{-1} \in H$.
        \end{enumerate}
    \end{prop}

    \begin{proof}
        \begin{itemize}
            \item ($\implies$) If $H$ is a subgroup of $G$, then $e_G\in H$, so $H\neq\emptyset$. Also, if $g,h\in H$, then $h^{-1} \in H$, so $gh^{-1} \in H$. \qqqedsym
            \item By (a), there exists $h\in H$, and by (b), $hh^{-1} = e_G\in H$. So by (b) again, $e_Gg^{-1} = g^{-1} \in H$ for every $g\in H$. Lastly, if $g,h\in H$, then $h^{-1} \in H$, so $gh = g\left( h^{-1}  \right) ^{-1} \in H$. \qqedsym
        \end{itemize} 
    \end{proof}

    \begin{prop}{Finite Subgroup Test}
        Let $G$ be a group and let $H\subseteq G$ be finite. Then $H$ is a subgroup of $G$ if and only if
        \begin{enumerate}
            \item $H\neq\emptyset$; and
            \item for every $g,h\in H, gh\in H$.
        \end{enumerate}
    \end{prop}

    \begin{proof}
        The forward direction is clear. Suppose $g\in H$. Since $H$ is finite, there must exist $j,k\in \N, j<k$ such that
        \begin{equation*}
            g^j = g^k.
        \end{equation*}
        Let $n=k-j$. Then observe that $g^n=e_G$ so $e_G\in H$ and
        \begin{equation*}
            g^{n-1} = g^{-1} \in H. \eqedsym
        \end{equation*}
    \end{proof}

    \section{Subgroups Generated by a Set}

    \begin{prop}{Intersection of Subgroups Is a Subgroup}
        Let $G$ be a group and let $\mF$ be a nonempty collection of subgroups of $G$. Then
        \begin{equation*}
            \bigcap^{}_{H\in\mF} H
        \end{equation*}
        is a subgroup of $G$.
    \end{prop}
    
    \begin{proof}
        For convenience denote the intersection by $K$. Since every $H\in\mF$ is a subgroup of $G$, so $e_G\in H$ for every $H\in\mF$. It follows that $e_G\in K$, so $k\neq\emptyset$. Now suppose that $g,h\in K$. Then
        \begin{flalign*}
            && \forall H\in\mF\left[ g,h\in H \right] & \implies \forall H\in\mF\left[ h^{-1} \in H \right]  && \\ 
            && & \implies \forall H\in\mF \left[ gh^{-1} \in H \right] && \\
            && & \implies gh^{-1} \in K.
        \end{flalign*} 
        Thus $K$ is a subgroup of $G$ by the subgroup test.
    \end{proof}
    
    \begin{definition}{Subgroup}{Generated by a Subset}
        Let $G$ be a group and let $S\subseteq G$. The \emph{subgroup generated by $S$} in $G$, denoted as $\left< S \right>$, is
        \begin{equation*}
            \left< S \right> = \bigcap^{}_{S\subseteq H\leq G} H, 
        \end{equation*}
        the intersection of every subgroup of $G$ containing $S$.\footnote{Note that $\left< S \right>$ is a subgroup by Proposition 2.4.}
    \end{definition} 

    \np[$\left< S \right>$]Observe that $\left< S \right>$ is the smallest subgroup containing $S$, since any subgroup $H\subseteq G$ that contains $S$ appears in the intersection. When we write out the elements of $S$, we often write
    \begin{equation*}
        \left< s_1,\ldots,s_k \right>  ,
    \end{equation*}
    where $S=\left\lbrace s_1,\ldots,s_k \right\rbrace$, instead of $\left< \left\lbrace s_1,\ldots,s_k \right\rbrace \right>$ for convenience.

    \np Previously we said that
    \begin{equation*}
        \left< s \right> = \left\lbrace s^k: k\in\left\lbrace 0,\ldots,n-1 \right\rbrace  \right\rbrace \leq D_{2n}.
    \end{equation*}
    First, $K = \left\lbrace s^k:k\in\left\lbrace 0,\ldots,n-1 \right\rbrace  \right\rbrace$ is a subgroup of $D_{2n}$ by the subgroup test, since $s\in K$ so $K\neq\emptyset$ and for every $g,h\in K, gh^{-1} = s^{l}$ for some $l\in\Z$, but
    \begin{equation*}
        K = \left\lbrace s^k:k\in\Z \right\rbrace 
    \end{equation*}
    since $s^n=e_{D_{2n}}$. But clearly any subgroup $H\leq D_{2n}$ that has $s$ must contain $K$, so $K = \left< s \right>$. Now we ask the following question: can we generalize this example to other cases? The answer is yes, and to see this, let us introduce the following notation: given any $S\subseteq G$, write $S^{-1} $ to dentoe
    \begin{equation*}
        S^{-1} = \left\lbrace s^{-1} :s\in S \right\rbrace .
    \end{equation*}

    \begin{prop}{}
        Let $G$ be a group and let $S\subseteq G$. Let
        \begin{equation*}
            K = \left\lbrace \prod^{k}_{j=1} s_j: k\in\N\cup\left\lbrace 0 \right\rbrace , s_1,\ldots,s_k\in S\cup S^{-1}  \right\rbrace .
        \end{equation*}
        Then $\left< S \right> = K$.\footnote{Observe that we are using the convention that the empty product is the identity.}
    \end{prop}

    \begin{proof}
        We have two claims.
        \begin{itemize}
            \item \textit{Claim 1}: \textit{$S\subseteq K\subseteq \left< S \right>$.}

                \begin{subproof}
                    We know that the empty product $e_G\in \left< S \right>$. The rest follows by induction.
                \end{subproof}

            \item \textit{Claim 2}: \textit{$K$ is a subgroup.}

                \begin{subproof}
                    Again, we know that the empty product $e_G\in \left< S \right>$. Suppose $g,h\in K$, where
                    \begin{equation*}
                        g = s_1\cdots s_k , h = t_1\cdots t_l
                    \end{equation*}
                    for some $k,l\in\N\cup\left\lbrace 0 \right\rbrace , s_1,\ldots,s_k,t_1,\ldots,t_l\in S\cup S^{-1} $. Then
                    \begin{equation*}
                        gh = s_1\cdots s_kt_1\cdots t_l\in K
                    \end{equation*}
                    by definition. Finally, $g^{-1} = s_k^{-1} \cdots s_1^{-1} \in K$ since $s_k^{-1} ,\ldots,s_1^{-1} \in S\cup S^{-1}$, since $S\cup S^{-1} $ is closed under taking inverses. Thus $K$ is a subgroup.\footnote{Note that we are directly using the definition of subgroup instead of the subgroup test here.} 
                \end{subproof}
        \end{itemize} 
        It is immediate from Claim 2 that $\left< S \right> \subseteq K$. Thus $K = \left< S \right>$, as desired.  
    \end{proof}

    \np[Lattice of Subgroups]Let $G$ be a group. Subgroups of $G$ are ordered by set inclusion $\subseteq$. If $H_1,H_2\leq G$ and $H_1\subseteq H_2$, then $H_1\leq H_2$, so we also write this order as $\leq$. The collection of subgroups of $G$ together with (partial) order $\leq$ is called the \emph{lattice of subgroups} of $G$. Given any $H_1,H_2\leq G$, the \emph{subgroup below} $H_1,H_2$ in the lattice is $H_1\cap H_2$. The \emph{subgroup above} $H_1,H_2$ is $\left< H_1\cup H_2 \right>$.  

    \section{Cyclic Groups}
    
    \begin{definition}{Generator}{of a Subgroup}
        Let $G$ be a group and let $H\leq G$. $S\subseteq G$ is called a \emph{generator} of $H$ if $\left< S \right> = H$. A group $G$ is \emph{cyclic} if $G=\left< g \right>$ for some $g\in G$.  
    \end{definition}

    \begin{prop}{}
        Let $G$ be a group and let $g\in G$.
        \begin{enumerate}
            \item $\left< g \right> = \left\lbrace g^j:j\in\Z \right\rbrace$. 
            \item If $\left| g \right| = n\in\N$, then $\left< g \right> = \left\lbrace g^j:j\in\left\lbrace 0,\ldots,n-1 \right\rbrace  \right\rbrace$. 
        \end{enumerate}
    \end{prop}

    \begin{prop}{}
        Let $G$ be a group and let $g\in G$. Then
        \begin{equation*}
            \left| \left< g \right> \right| = \left| g \right| .
        \end{equation*}
    \end{prop}

    \begin{proof}
        Denote $H = \left< g \right>$ for convenience. We already know that $\left| H \right| \leq \left| g \right|$ from Proposition 2.6. Suppose that $\left| G \right| = n\in\N$ without loss of generality. This means 
        \begin{equation*}
            a^0, \ldots, a^n
        \end{equation*}
        must have repetition: there are $j,k\in \left\lbrace 0,\ldots,n \right\rbrace$ with $a^j=a^k$. This means $a^{k-j} = e_G$, so $\left| g \right| \leq n$, implying $\left| g \right| \leq \left| G \right|$. Thus $\left| g \right| = \left| G \right|$, as desired.
    \end{proof}
    
    \begin{prop}{}
        Let $G$ be a group and let $S\subseteq G$ be such that $\left< S \right> = G$. Then, given any $T\subseteq G$,
        \begin{equation*}
            G = \left< T \right>  
        \end{equation*}
        if and only if $S\subseteq \left< T \right>$. 
    \end{prop}

    \np[Generators of $\Z /n\Z$]Let $n\in\N$ for the rest of this section, and consider the group $\Z /n\Z$. We know that $\Z /n\Z = \left< \left[ 1 \right]  \right>$, so by Proposition 2.8, $\left< \left[ a \right]  \right> = \Z /n\Z$ if and only if $\left[ 1 \right] \in \left< \left[ a \right]  \right>$, given any $a\in\Z$. But
    \begin{equation*}
        \left[ 1 \right] \in \left< \left[ a \right]  \right> \iff \gcd\left( a,n \right) = 1, 
    \end{equation*}
    so $\left< \left[ a \right]  \right> = \Z /n\Z$ if and only if $\gcd\left( a,n \right) = 1$ (i.e. $a$ is coprime to $n$). 

    \clearpage
    \begin{prop}{}
        Let $G$ be a group and let $g\in G$ be such that $g^n=e_G$. Then $\left.\left| g \right|\right| n$. 
    \end{prop}
    
    \begin{prop}{}
        Suppose $a|n$, where $a\in\N$. Then $\left| \left[ a \right]  \right| = \frac{n}{a}$.
    \end{prop}

    \begin{proof}
        Let $k\in\N$ be such that $n=ka$. Then $l\left[ a \right] \neq 0$ for all $l\in\left\lbrace 1,\ldots,k-1 \right\rbrace$. But
        \begin{equation*}
            k\left[ a \right] = \left[ ka \right] = 0
        \end{equation*}
        so the order of $\left[ a \right]$ is $k$.
    \end{proof}

    \begin{prop}{}
        Let $a\in\Z$ and let $b=\gcd\left( a,n \right)$. Then $\left< \left[ a \right]  \right> = \left< \left[ b \right]  \right>$.  
    \end{prop}

    \begin{proof}
        Since $b|a$, there exists $k\in\Z$ such that $a=kb$. This means $\left[ a \right] \in \left< \left[ b \right]  \right>$, implying $\left< \left[ a \right]  \right> \subseteq \left< \left[ b \right]  \right>$. But by Bezout's lemma, there exist $x,y\in\Z$ such that
        \begin{equation*}
            xa+yn = b.
        \end{equation*}
        So $\left[ b \right] = x\left[ a \right]$, implying $\left< \left[ b \right]  \right> \subseteq \left< \left[ a \right]  \right>$. Thus the result is established.  
    \end{proof}
    
    \begin{prop}{}
        Suppose $a\in\Z$. Then
        \begin{equation*}
            \left| \left[ a \right]  \right| = \frac{n}{\gcd\left( a,n \right) }. 
        \end{equation*}
    \end{prop}

    \begin{proof}
        Let $b=\gcd\left( a,n \right)$. Then $\left< \left[ a \right]  \right> = \left< \left[ b \right]  \right> $, so
        \begin{equation*}
            \left| \left[ a \right]  \right| = \left| \left[ b \right]   \right| = \frac{n}{b} = \frac{n}{\gcd\left( a,n \right) }. \eqedsym,
        \end{equation*}
    \end{proof}

    \begin{cor}{}
        Let $n\in\N$. 
        \begin{enumerate}
            \item The order of any cyclic subgroup of $\Z /n\Z$ divides $n$.
            \item For every $d\in\N$ that divides $n$, there exists a unique cyclic subgroup of $\Z /n\Z$ of order $d$, generated by $\left[ \frac{n}{d} \right]$.
        \end{enumerate}
    \end{cor}	

    \begin{proof}
        \begin{enumerate}
            \item This follows immediately from Proposition 2.12. \qqqedsym
            \item For the existence part, first note that $\left| \left[ \frac{n}{d} \right]  \right| = d$ by Proposition 2.10. So $\left< \left[ \frac{n}{d} \right]  \right>$ is a cyclic subgroup of $\Z /n\Z$ of order $d$. Moreover, given any $a\in\Z$ such that $\left< \left[ a \right]  \right> = d$, by Proposition 2.12,
                \begin{equation*}
                    \gcd\left( a,n \right) = \frac{n}{d},
                \end{equation*}
                so by Proposition 2.11 $\left< \left[ a \right]  \right> = \left< \left[ \frac{n}{d} \right]  \right>$. \qqedsym  
        \end{enumerate}
    \end{proof}
    
    \clearpage
    \section{Homomorphisms}

    \begin{definition}{Homomorphism}{between Groups}
        Let $G,H$ be groups. A function $\varphi:G\to H$ such that
        \begin{equation*}
            \varphi\left( gh \right) = \varphi\left( g \right) \varphi\left( h \right) 
        \end{equation*}
        for all $g,h\in G$ is called a (group) \emph{homomorphism}.
    \end{definition}

    \np Intuitively speaking, group homomorphisms are functions that \textit{preserve the group structure}.

    \begin{prop}{Properties of Homomorphisms}
        Let $G,H$ be groups and let $\varphi:G\to H$ be a homomorphism.
        \begin{enumerate}
            \item $\varphi\left( e_G \right) = e_H$.
            \item $\varphi\left( g^n \right) = \varphi\left( g \right) ^n$ for all $g\in G, n\in\Z$.
            \item $\left| \varphi\left( g \right)  \right| | \left| g \right|$ for all $g\in G$.\footnote{We are following the convention that $n|\infty$ for all $n\in\N$.}
        \end{enumerate}
    \end{prop}

    \begin{proof}
        \begin{enumerate}
            \item Observe that
                \begin{equation*}
                    \varphi\left( e_G \right) = \varphi\left( e_Ge_G \right) = \varphi\left( e_G \right) \varphi\left( e_G \right) 
                \end{equation*}
                so $\varphi\left( e_G \right) = e_H$. \qqqedsym
            \item The result is clear for $n\in\N$ by induction. The case $n=0$ follows from (a). So it suffices to show that
                \begin{equation*}
                    \varphi\left( g^{-1}  \right) = \varphi\left( g \right) ^{-1} .
                \end{equation*}
                But note that
                \begin{equation*}
                    \varphi\left( g^{-1}  \right) \varphi\left( g \right) = \varphi\left( g^{-1} g \right) = \varphi\left( e_G \right) = e_H,
                \end{equation*}
                so it follows that $\varphi\left( g^{-1}  \right) = \varphi\left( g \right) ^{-1}$. \qqqedsym
            \item Without loss of generality assume that $g$ has a finite order, say $\left| g \right| = n\in\N$. Then 
                \begin{equation*}
                    \varphi\left( g \right) ^n = \varphi\left( g^n \right) = \varphi\left( e_G \right) = e_H
                \end{equation*}
                by (a), (b), so it follows that $\left| \varphi\left( g \right)  \right|$ is at most $g$. But if $\left| \varphi\left( g \right)  \right|$ does not divide $n$, then $\varphi\left( g \right) ^n = \varphi\left( g \right) ^r$ for some $r<\left| \varphi\left( g \right)  \right|$, implying that $\varphi\left( g \right) ^n\neq e_H$, which is a contradiction. Thus $\left| \varphi\left( g \right)  \right|$ divides $n$. \qqedsym
        \end{enumerate}
    \end{proof}

    \begin{prop}{}
        Let $G$ be a group and let $K\leq G$ with the induced operation from $G$. Then $\eta:K\to G$ by
        \begin{equation*}
            \eta\left( g \right) = g
        \end{equation*}
        for every $g\in K$ is a homomorphism.
    \end{prop}

    \begin{proof}
        Observe that
        \begin{equation*}
            \varphi\left( gh \right) = gh = \varphi\left( g \right) \varphi\left( h \right) 
        \end{equation*}
        for every $g,h\in H$.
    \end{proof}

    \begin{prop}{Composition of Homomorphisms}
        Let $G,H,K$ be groups and let $\varphi:G\to H, \psi:H\to K$ be homomorphisms. Then $\psi\circ\varphi:G\to K$ is a homomorphism.
    \end{prop}

    \begin{proof}
        Let $g,h\in G$. Then
        \begin{equation*}
            \psi\left( \varphi\left( gh \right)  \right) = \psi\left( \varphi\left( g \right) \varphi\left( h \right)  \right) = \psi\left( \varphi\left( g \right)  \right) \psi\left( \varphi\left( h \right)  \right) . \eqedsym
        \end{equation*}
    \end{proof}

    \begin{cor}{}
        Let $G,H$ be groups and let $K\leq G$. Let $\varphi:G\to H$ be a homomorphism. Then $\varphi|_K:K\to H$ is a homomorphism.
    \end{cor}	

    \begin{proof}
        Let $g,h\in K$. Then
        \begin{equation*}
            \varphi|_K\left( gh \right) = \varphi\left( gh \right) = \varphi\left( g \right) \varphi\left( h \right) = \varphi|_K\left( g \right) \varphi|_K\left( h \right) .
        \end{equation*}
        Alternatively, observe that $\varphi|_K = \varphi\circ\eta$, where $\eta$ is the homomorphism discussed in Proposition 2.14.
    \end{proof}
    
    \begin{definition}{Restriction}{of a Homomorphism to a Subgroup}
        Consider the setting of Corollary 2.15.1. $\varphi|_K$ is called the \emph{restriction} (homomorphism) of $\varphi$ to $K$.
    \end{definition}

    \begin{prop}{}
        Let $G,H$ be groups and let $\varphi:G\to H$ be a homomorphism. Let $K\leq G$. Then
        \begin{equation*}
            \varphi\left( K \right) \leq H.
        \end{equation*}
    \end{prop}

    \begin{proof}
        Since $K\neq\emptyset$, $\varphi\left( K \right) \neq\emptyset$. Moreover, given any $x,y\in\varphi\left( K \right)$, there are $g,h\in K$ such that
        \begin{equation*}
            x = \varphi\left( g \right) , y=\varphi\left( h \right) .
        \end{equation*}
        So $xy^{-1} = \varphi\left( g \right) \varphi\left( h \right) ^{-1} = \varphi\left( gh^{-1}  \right) \in \varphi\left( K \right)$. Thus by the subgroup test the result follows.
    \end{proof}

    \begin{definition}{Image}{of a Homomorphism}
        Let $G,H$ be groups and let $\varphi:G\to H$ be a homomorphism. Then the \emph{image} of $\varphi$, often denoted as $\image\left( \varphi \right)$, is the subgroup $\varphi\left( G \right)$ of $H$.
    \end{definition}

    \begin{prop}{}
        Let $G,H$ be groups and let $\varphi:G\to H$ be a homomorphism such that $\varphi\left( G \right) \leq K$ for some $K\leq H$. Then $\tilde{\varphi}:G\to K$ defined by
        \begin{equation*}
            \tilde{\varphi} \left( g \right) = \varphi\left( g \right) 
        \end{equation*}
        is also a homomorphism with $\tilde{\varphi}\left( G \right) = \varphi\left( G \right)$.
    \end{prop}

    \begin{proof}
        Observe that, given any $g,h\in G$,
        \begin{equation*}
            \tilde{\varphi} \left( gh \right) = \varphi\left( gh \right) = \varphi\left( g \right) \varphi\left( h \right) = \tilde{\varphi} \left( g \right) \tilde{\varphi} \left( h \right) .
        \end{equation*}
        Moreover, it is by definition that $\tilde{\varphi} \left( G \right) = \varphi\left( G \right)$.
    \end{proof}

    \clearpage
    \begin{cor}{}
        Let $G,H$ be groups. Then any homomorphism $\varphi:G\to H$ induces a surjective homomorphism $\tilde{\varphi}:G\to K$, where $K=\varphi\left( G \right)$.\footnote{We usually refer this surjective homomorphism $\tilde{\varphi}$ as $\varphi$ for convenience.} 
    \end{cor}	
    
    \begin{prop}{}
        Let $G,H$ be groups and let $\varphi:G\to H$ be a homomorphism. Then 
        \begin{equation*}
            \varphi\left( \left< S \right>   \right) = \left< \varphi\left( S \right)  \right>  
        \end{equation*}
        for every $S\subseteq G$.
    \end{prop}

    \begin{proof}
        Observe that
        \begin{equation*}
            \varphi\left( S^{-1}  \right) = \left\lbrace \varphi\left( s^{-1}  \right) :s\in S \right\rbrace = \left\lbrace \varphi\left( s \right) ^{-1} :s\in S \right\rbrace = \varphi\left( S \right) ^{-1} .
        \end{equation*}
        So
        \begin{flalign*}
            && \varphi\left( \left< S \right>   \right) & = \varphi\left( \left\lbrace \prod^{k}_{j=0} s_j: k\in\N\cup\left\lbrace 0 \right\rbrace , s_1,\ldots,s_k\in S\cup S^{-1}  \right\rbrace  \right)  && \\ 
            && & = \left\lbrace \prod^{k}_{j=0} \varphi\left( s_j \right) : k\in\N\cup\left\lbrace 0 \right\rbrace , s_1,\ldots,s_k\in S\cup S^{-1}  \right\rbrace && \\
            && & = \left\lbrace \prod^{k}_{j=0} t_j: k\in\N\cup\left\lbrace 0 \right\rbrace , t_1,\ldots.,t_k\in \varphi\left( S \right) \cup \varphi\left( S \right) ^{-1}  \right\rbrace && \\
            && & = \left< \varphi\left( S \right)  \right> .  && \fqedsym
        \end{flalign*} 
    \end{proof}

    \begin{prop}{Preimage of a Subgroup under a Homomorphism Is a Subgroup}
        Let $G,H$ be groups and let $\varphi:G\to H$ be a homomorphism. Then for every $K\leq H$,
        \begin{equation*}
            \varphi^{-1} \left( K \right) \leq G.
        \end{equation*}
    \end{prop}

    \begin{proof}
        Observe that $\varphi\left( e_G \right) = e_H \in K$, so $e_G\in\varphi^{-1} \left( K \right)$. Moreover, given any $g,h\in\varphi^{-1} \left( K \right)$, $\varphi\left( g \right) , \varphi\left( h \right) \in K$, so
        \begin{equation*}
            \varphi\left( gh^{-1}  \right) = \varphi\left( g \right) \varphi\left( h \right) ^{-1} \in K.
        \end{equation*}
        Thus $gh^{-1} \in \varphi^{-1} \left( K \right)$.
    \end{proof}

    \begin{definition}{Kernel}{of a Homomorphism}
        Let $G,H$ be groups and let $\varphi:G\to H$ be a homomorphism. Then the \emph{kernel} of $\varphi$, denoted as $\ker\left( \varphi \right)$, is the subgroup
        \begin{equation*}
            \ker\left( \varphi \right) = \varphi^{-1} \left( \left\lbrace e_H \right\rbrace  \right)\leq G. 
        \end{equation*}
    \end{definition}

    \begin{prop}{Characterization of Injectivity of a Homomorphism}
        Let $G,H$ be groups and let $\varphi:G\to H$ be a homomorphism. Then $\varphi$ is injective if and only if $\ker\left( \varphi \right) = \left\lbrace e_H \right\rbrace $.
    \end{prop}

    \begin{proof}
        \begin{itemize}
            \item ($\implies$) Since $\varphi$ is a homomorphism, $\varphi\left( e_G \right) = e_H$, so if $\varphi$ is injective, given any $g\in G$, $\varphi\left( g \right) = e_H$ if and only if $g=e_G$. Thus $\ker\left( \varphi \right) = \left\lbrace e_G \right\rbrace$. \qqqedsym
            \item Suppose that $\ker\left( \varphi \right) = \left\lbrace e_G \right\rbrace$, and let $g,h\in G$ be such that
                \begin{equation*}
                    \varphi\left( g \right) = \varphi\left( h \right) ,
                \end{equation*}
                where it suffices to show $g=h$. Then
                \begin{equation*}
                    \varphi\left( gh^{-1}  \right) = \varphi\left( g \right) \varphi\left( h \right) ^{-1} = e_H,
                \end{equation*}
                so $gh^{-1} \in \ker\left( \varphi \right)$. This means $gh^{-1} = e_G$, so $g=h$ as desired. \qqedsym
        \end{itemize} 
    \end{proof}

    \begin{prop}{Any Subgroup of a Cyclic Group Is Cyclic}
        Let $G$ be a cyclic. Then for any $H\leq G$, $H$ is cyclic.
    \end{prop}

    \begin{proof}
        We have the following three claims.
        \begin{itemize}
            \item \textit{Claim 1}: \textit{All subgroups of $\Z$ are of the form $m\Z = \left< m \right>$, hence cyclic.}
            \item \textit{Claim 2}: \textit{Given any group $K$, $K$ is cyclic if and only if there is a surjective homomorphism $\Z\to K$.}
            \item \textit{Claim 3}: \textit{If $X,Y$ are sets, $f:X\to Y$ is a surjection, and $S\subseteq Y$ , then $f\left( f^{-1} \left( S \right)  \right)= S$.}
        \end{itemize}
        Since $G$ is cyclic, there is a surjective homomorphism $\varphi:\Z\to G$ by Claim 2. Since all subgroups of $\Z$ are cyclic by Claim 1, there is $m\in\Z$ such that $\varphi^{-1} \left( H \right) = \left< m \right>$. Let $\psi:\Z\to\Z$ be defined by
        \begin{equation*}
            \varphi\left( k \right) = mk
        \end{equation*}
        for every $k\in\Z$, which is a homomorphism. Then $\varphi\circ\psi:\Z\to G$ is a surjective homomorphism with
        \begin{equation*}
            \varphi\left( \psi\left( \Z \right)  \right) = \varphi\left( m\Z \right) = \varphi\left( \varphi^{-1} \left( H \right)  \right)= H
        \end{equation*}
        by Claim 3. Therefore, by restricting the codomain of $\varphi\circ\psi$ to get a surjective homomorphism from $\Z$ to $H$. Thus by Claim 2 $H$ is cyclic.
    \end{proof}

    \section{Isomorphisms}

    \begin{definition}{Isomorphism}{between Groups}
        Let $G,H$ be groups and let $\varphi:G\to H$ be a homomorphism. If $\varphi$ is bijective, then we say $\varphi$ is an \emph{isomorphism}.
    \end{definition}

    \begin{prop}{Inverse of an Isomorphism Is an Isomorphism}
        Let $G,H$ be groups and let $\varphi:G\to H$ be an isomorphism. Then $\varphi^{-1} :H\to G$ is also an isomorphism.
    \end{prop}

    \begin{proof}
        Since $\varphi^{-1} $ is bijective, it suffices to show that $\varphi^{-1} $ is a homomorphism. Let $g,h\in H$. Then
        \begin{equation*}
            \varphi\left( \varphi^{-1} \left( g \right) \varphi^{-1} \left( h \right)  \right) = \varphi\left( \varphi^{-1} \left( g \right)  \right) \varphi\left( \varphi^{-1} \left( h \right)  \right) = gh,
        \end{equation*}
        which means
        \begin{equation*}
            \varphi^{-1} \left( g \right) \varphi^{-1} \left( h \right) = \varphi^{-1} \left( gh \right) ,
        \end{equation*}
        so $\varphi^{-1} $ is homomorphic.
    \end{proof}

    \clearpage
    \begin{cor}{}
        Let $G,H$ be groups and let $\varphi:G\to H$ be a homomorphism. Then $\varphi$ is an isomorphism if and only if there exists a homomorphism $\psi:H\to G$ such that
        \begin{enumerate}
            \item $\psi\circ\varphi = e_G$; and
            \item $\varphi\circ\varphi = e_H$.
        \end{enumerate}
    \end{cor}	

    \begin{definition}{Isomorphic}{Groups}
        Let $G,H$ be groups. If there exists an isomorphism between $G,H$, then we say $G,H$ are \emph{isomorphic}, denoted as $G\iso H$.
    \end{definition}

    \np $\iso$ is an equivalence relation. That is, given groups $G,H,K$,
    \begin{enumerate}
        \item $G\iso H\implies H\iso G$;
        \item $G\iso H, H\iso K\implies G\iso K$; and
        \item $G\iso G$.
    \end{enumerate}

    \begin{prop}{}
        Let $G,H$ be isomorphic groups. Then
        \begin{enumerate}
            \item $\left| G \right| = \left| H \right| $;
            \item $G$ is abelian if and only if $H$ is abelian;
            \item $\left| g \right| = \left| \varphi\left( g \right)  \right| $ for all $g\in G$; and
            \item $K\subseteq G$ is a subgroup of $G$ if and only if $\varphi\left( K \right) $ is a subgroup of $H$.
        \end{enumerate}
    \end{prop}

    \begin{prop}{Characterization of Isomorphic Cyclic Groups}
        Let $G,H$ be cyclic groups. Then $G\iso H$ if and only if $\left| G \right| = \left| H \right| $.
    \end{prop}
    
    \begin{proof}
        Let $g\in G, h\in H$ be such that $G=\left< g \right> , H = \left< h \right>$.   
        \begin{itemize}
            \item ($\implies$) This direction is clear. \qqqedsym
            \item ($\impliedby$) Suppose that $\left| G \right| = \left| H \right|$. Define $\varphi:G\to H$ so that
                \begin{equation*}
                    \varphi\left( g^j \right) = b^j
                \end{equation*}
                for every $j\in\Z$. Note that this $\varphi$ is well-defined, since if $g^j = g^k$ for some $j,k\in\Z$, then $\left| g \right| | k-j$, so $\left| h \right| | k-j$, since we assumed $\left| G \right| = \left| H \right| $ so $\left| g \right| = \left| h \right|$. But this means $h^j = h^k$. To show that $\varphi$ is homomorphic, let $j,k\in\Z$. Then
                \begin{equation*}
                    \varphi\left( g^jg^k \right) = \varphi\left( g^{j+k} \right) = h^{j+k} = h^jh^k = \varphi\left( g^j \right) \varphi\left( g^k \right) .
                \end{equation*}
                To show that $\varphi$ is invertible, observe that we can define $\psi: H\to G$ such that
                \begin{equation*}
                    \psi\left( h^j \right) = g^j
                \end{equation*}
                for every $j\in\Z$, where this $\psi$ is a well-defined homomorphism via similar arguments, and it is clear that $\psi$ is the inverse of $\varphi$. \qqedsym
        \end{itemize} 
    \end{proof}
    
    \clearpage
    \begin{cor}{Identification of Cyclic Groups}
        Let $G$ be a cyclic group.
        \begin{enumerate}
            \item If $\left| G \right| = \infty$, then $G\iso\Z$.
            \item If $\left| G \right| = n\in\N$, then $G\iso \Z /n\Z$.
        \end{enumerate}
    \end{cor}	

    \begin{cor}{}
        Every cyclic group is abelian.
    \end{cor}	

    \begin{definition}{$C_{n}$}{}
        Let $a$ be a formal indeterminate. We define
        \begin{enumerate}
            \item $C_{\infty} = \left\lbrace a^k:k\in\Z \right\rbrace$, where $a^ja^k = a^{j+k}$ for every $j,k\in\Z$; and
            \item $C_n = \left\lbrace a^k: k\in \Z /n\Z \right\rbrace$, where $a^ja^k=a^{j+k}$ for every $j,k\in\Z /n\Z$.
        \end{enumerate}
    \end{definition}

    \np It is immediate that $C_n$'s are groups under the given operations. $C_n$'s are useful when we want to use multiplicative notation for cyclic groups. Note that
    \begin{equation*}
        C_{\infty} \iso \Z, C_n\iso \Z/n\Z
    \end{equation*}
    for every $n\in\N$.
    
    
    
    
    
    
    
    
    
    
    
    
    
    
    
    
    
    
    
    
    
    
    
    
    
    
    
    
    
    
    
    
    
    
    

\end{document}
